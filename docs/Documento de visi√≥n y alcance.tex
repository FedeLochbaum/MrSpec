\documentclass{article}
\usepackage[utf8]{inputenc}
\usepackage[spanish]{babel}

\title{Documento de visión y alcance}
\author{Mr. Spec \\ \\ \\ \\}
\date{
\begin{tabular}{ | c | c | }
 \hline Lochbaum, Federico & federico.lochbaum@gmail.com \\
 \hline López, Sebastián & sebastianariell@gmail.com \\ 
 \hline Papadopulo, Rodrigo & rpapadopulo.2106@gmail.com \\
 \hline
 \end{tabular}}

\begin{document}

\maketitle
\thispagestyle{empty} 

\newpage
\pagenumbering{arabic}
\tableofcontents

\newpage

\section{Introducción}

Este documento tiene como objetivo documentar la visión y alcance del proyecto.\newline

El objetivo es proveer una visión de alto nivel de los elementos que se abarcarán en el sistema, los que no se incluirán,
que funcionalidad se incluira, sus necesidades, los objetivos principales del sistema y las restricciones.

\newpage
\section{Enunciado del problema}
\subsection{Necesidad}

Hay varias plataformas que utilizan BDD para realizar testing pero ninguna que anuncie explícitamente el principio 
Given-When-Then, por lo tanto hemos considerado necesario desarrollar una herramienta que posibilite esto, de forma 
que se pueden desarrollar proyectos utilizando BDD con lenguaje "coloquial".

\subsection{Problemas \& Motivación}

Debido a la falta de frameworks de testing que permitan realizar BDD utilizando Given - When - Then se decidió desarrollar 
en Ruby y para Ruby un framework capaz de presentar tests con estas características.

\newpage

\section{Solución Propuesta}

\subsection{Objetivos}

El framework a desarrollar le permitirá a los usuarios que lo utilicen poder crear y correr tests con diferentes aserciones 
utilizando el principio Given-When-Tthen. \newline

Además podrán ver informes detallados según si el test fue correcto o incorrecto.

\subsection{Alcance}

El sistema a desarrollar permitirá crear y correr distintos tipos de tests para comprobar la funcionalidad de un sistema.

\subsection{Stakeholders \& Usuarios}
\subsubsection{Perfiles de los Stakeholders}
\begin{itemize}
    \item Usuarios de Ruby
\end{itemize}
\subsubsection{Perfiles de los Usuarios}
\begin{itemize}
    \item Personas que utilicen Ruby
    \item Interesados en realizar BDD en sus proyectos
\end{itemize}
\newpage

\section{Cronograma}
\newpage

\end{document}
